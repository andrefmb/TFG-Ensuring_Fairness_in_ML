\chapter*{Resumen}

% Los artículos y libros incluidos en el archivo research.bib pueden
% citarse desde cualquier punto del texto usando ~\cite.

Los modelos de aprendizaje automático tienen un impacto cada vez mayor en el mundo actual, siendo utilizados para asistir y a veces sustituir a los humanos en muchos entornos. Estos modelos, a menudo funcionan aprendiendo sobre decisiones históricas tomadas por distintos grupos sociales con diferentes tasas de error en su clasificación, por lo que surge la necesidad de vigilar y controlar el sesgo involuntario de los modelos predictivos contra estos
grupos de población desfavorecidos.

La mayoría de las herramientas utilizadas para mitigar el sesgo entre los grupos no privilegiados suelen depender del modelo y la métrica utilizada, aumentando así los requerimientos de los procesos en las técnicas de aprendizaje automático empleadas. Buscaremos entonces un concepto de equidad que nos permita eliminar en mayor medida la desigualdad de rendimiento entre grupos, pero que a su vez no aumente el nivel de complejidad del modelo.

En este trabajo, formalizaremos matemáticamente las definiciones de diferentes medidas de justicia y
equidad; presentaremos sus propiedades, limitaciones e incompatibilidades e indagaremos en las posibles opciones
para mejorar los resultados obtenidos mediante un proceso de aprendizaje automático en los
términos planteados previamente:
equidad por desconocimiento, paridad estadística/demográfica, medidas causales, equidad individual, entre otras.

Nos enfocaremos en desarrollar un marco para modelar la equidad usando herramientas de inferencia causal tomando como base la teoría de probabilidad y estadística. Discutiendo el concepto de equidad contrafactual que plantea que una decisión es justa para un individuo si es la misma en el mundo real y en un
mundo ''contrafactual'' en el que el individuo perteneciese a un grupo demográfico
diferente. 

Finalmente haremos un análisis de las utilidades que ofrece el software Aequitas para garantizar justicia en aprendizaje automático. Realizaremos algunas pruebas para la evaluación de las medidas de equidad definidas y lo complementaremos con la resolución de un problema del mundo real utilizando herramientas de equidad contrafactual en Python. En este contexto, mostraremos herramientas de presentación y visualización de resultados que faciliten una explicación de la causa del sesgo y ayuden a los usuarios a tomar decisiones en el mundo real.

\paragraph{Palabras clave:} medidas de equidad, mitigación del sesgo, impacto dispar, teorema de imposibilidad, equidad contrafactual, Aequitas.