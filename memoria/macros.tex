%------------------------
% Bibliotecas para matemáticas de latex
%------------------------
\usepackage{amsthm}
\usepackage{amsmath}
\usepackage[ruled, spanish, onelanguage]{algorithm2e}
\usepackage{tikz}
\usepackage{tikz-cd}
\usetikzlibrary{shapes, fit, automata,  positioning, arrows}
\usepackage{bussproofs}
\EnableBpAbbreviations{}
\usepackage{mathtools}
\usepackage{scalerel}
\usepackage{verbatim} % comentarios
\usepackage{stmaryrd}
\usepackage{natbib}
\usepackage{bm}
\usepackage{amsthm}
\usepackage[theorems, skins, breakable]{tcolorbox}

% Licencia
\usepackage[
    type={CC},
    modifier={by-sa},
    version={4.0},
]{doclicense}

% Glossaries
%\usepackage[toc, nopostdot,  style=super, nonumberlist, section=chapter ]{glossaries}
%\newglossary{symbols}{sym}{sbl}{List of Abbreviations and Symbols}
%\makeglossaries
%\loadglsentries{bibliography/glossary}

%------------------------
% Estilos para los teoremas
%------------------------
\theoremstyle{definition}
\newtheorem{theorem}{Teorema}
\newtheorem{proposition}{Proposición}
\newtheorem{lemma}{Lema}
\newtheorem{corollary}{Corolario}
\newtheorem{definition}{Definición}
\theoremstyle{remark}
\newtheorem{remark}{Comentario}
\newtheorem{example}{Ejemplo}
\newtheorem{result}{Resultado}
\theoremstyle{definition}
\newtheorem{notation}{Notación}

\tcolorboxenvironment{definition}{
  blanker,
  breakable,
  left=12pt,
  %before skip=12pt,
  after skip=12pt,
  borderline west={2pt}{0pt},
  before upper={\parindent 12pt}
}

\tcolorboxenvironment{proposition}{
  blanker,
  breakable,
  left=12pt,
  %before skip=12pt,
  after skip=12pt,
  borderline west={2pt}{0pt},
  before upper={\parindent 12pt}
}

\tcolorboxenvironment{theorem}{
  blanker,
  breakable,
  left=12pt,
  %before skip=12pt,
  after skip=12pt,
  borderline west={2pt}{0pt}{red},
  before upper={\parindent 12pt}
}

\tcolorboxenvironment{corollary}{
  blanker,
  breakable,
  left=12pt,
  %before skip=12pt,
  after skip=12pt,
  borderline west={2pt}{0pt}{red},
  before upper={\parindent 12pt}
}

\tcolorboxenvironment{lemma}{
  blanker,
  breakable,
  left=12pt,
  %before skip=12pt,
  after skip=12pt,
  borderline west={2pt}{0pt}{red},
  before upper={\parindent 12pt}
}

\begingroup\makeatletter\@for\theoremstyle:=definition,remark,lemma,plain\do{\expandafter\g@addto@macro\csname th@\theoremstyle\endcsname{\addtolength\thm@preskip\parskip}}\endgroup

%------------------------
% Macros
% ------------------------

% Aquí pueden añadirse abreviaturas para comandos de latex
% frequentemente usados.
\newcommand*\diff{\mathop{}\!\mathrm{d}}
\newcommand\ddfrac[2]{\frac{\displaystyle #1}{\displaystyle #2}}
\newcommand{\abs}[1]{\left\lvert#1\right\rvert} 
\newcommand{\norm}[1]{\left\lVert#1\right\rVert}
\newcommand{\norminf}[1]{\left \| #1 \right \|_\infty} 
\newcommand{\KN}{\mathbb K^{\mathbb N}}
\newcommand{\RN}{\mathbb R^{\mathbb N}}
\newcommand{\K}{\mathbb K}
\newcommand{\N}{\mathbb N}
\newcommand{\R}{\mathbb{R}}
\newcommand{\D}{\ \mathrm d}
\newcommand{\Ro}{\R_0^+}
\newcommand{\M}{\mathcal{M}}
\newcommand{\PR}{\mathcal{P}(\R^N)}
\newcommand{\mst}{(X_1,\dots,X_n)}
\newcommand{\htheta}{\bm{\hat{\theta}}}
\newcommand{\Rn}{\R^N}
\newcommand{\nm}[1]{ \textbf{#1}}
\newcommand{\btheta}{\bm{\theta}}
\newcommand{\bigCI}{\mathrel{\text{\scalebox{1.07}{$\perp\mkern-10mu\perp$}}}}
\newcommand\restr[2]{{% we make the whole thing an ordinary symbol
  \left.\kern-\nulldelimiterspace % automatically resize the bar with \right
  #1 % the function
  \vphantom{\big|} % pretend it's a little taller at normal size
  \right|_{#2} % this is the delimiter
  }}
\newcommand\mm[3][]{\begin{tabular}{@{}c@{}}
  \ensuremath{\textbf{#2}}\\[-1.2ex]{\text{\tiny(#3)}}
  \end{tabular}}
